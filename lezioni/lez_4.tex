\lez{4}{02-03-2020}{}
\[
	\frac{\mbox{d} I_{\nu} }{\mbox{d} s} = j _{\nu} -\alpha _{\nu} \quad d\tau _{\nu} = \alpha _{\nu} ds \text{ Profondità ottica}
.\] 
\[
	\frac{\mbox{d} I_{\nu} }{\mbox{d} \tau _{\nu} } = s_{\nu} - I_{\nu} \quad S_{\nu} = \frac{j _{\nu} }{\alpha _{\nu} } \text{S: funzione sorgente}
.\] 
Soluzione formale eq. trasporto
\[
	I_{\nu} ( \tau _{\nu} ) = I_{\nu} ( 0) e^{-\tau _{\nu} } + \int_{0}^{\infty} s_{\nu} ( \tau _{\nu} ) e^{-\left( \tau _{\nu} -\tau _{\nu} ' \right) }
	\text{ messo omogeneo}
.\] 
Caso di mezzo omogeneo:
\[
	I_{\nu} ( \tau _{\nu} ) = I_{\nu} ( 0) e^{-\tau _{\nu} } + S_{\nu} \left( 1- e^{-\tau _{\nu} } \right) 
.\] 
\[
	\text{Se } \tau _{\nu} \gg 1 \implies I_{\nu} ( \tau _{\nu} ) \to S_{\nu} 
.\] 
Altrimenti
\[
	I_{\nu} ( \tau _{\nu} )  \approx j _{\nu} L
.\] 
\subsection{Scoperta delle righe}%
Le righe sono state scoperte osservando lo spettro del sole, 


