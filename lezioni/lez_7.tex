\lez{7}{16-03-2020}{}
\subsection{Studio delle righe di assorbimento}
\label{subsec:Studio delle righe di assorbimento}
\subsubsection{Informazioni che ci arrivano dalle righe.}
\label{subsubsec:Informazioni che ci arrivano dalle righe.}
L'identificazione delle righe negli spettri stellari è molto importante, infatti dalla conoscenza della lunghezza d'onda centrale della riga possiamo conoscere:
\begin{itemize}
	\item L'atomo che le ha causate assorbendo fotoni.
	\item Lo stato energetico dell'atomo (perchè una data lunghezza d'onda corrisponde ad una determinata transizione energetica).
	\item Lo stato di ionizzazione dell'atomo: atomi dello stesso elemento chimico ma in stati di ionizzazione differenti hanno livelli energetici differenti e quindi tranzizioni differenti.
	\item La temperatura dell'atmosfera di quell'elemento: il popolamento dei livelli energetici associati alla transizione incriminata dipenderà dalla temperatura.
	\item La densità di quell'elemento nella atmosfera.
\end{itemize}
Naturalmente le ultime due informazioni citate, che riguardano la condizione fisica della atmosfera in cui l'atomo è immerso, riguardano esclusivamente la fotosfera della stella se facciamo osservazione nel visibile.\\
Un'altro parametro che può darci molte informazioni è la larghezza della riga, vediamo un esempio di riga nello spettro per avere un pò di nomenclatura di riferimento:
\begin{figure}[H]
    \centering
    \incfig{riga-di-assorbimento-generica-nello-spettro}
    \caption{Riga di assorbimento generica nello spettro.}
    \label{fig:riga-di-assorbimento-generica-nello-spettro}
\end{figure}
\noindent
\subsubsection{Allargamento di una riga.}
\label{subsubsec:Allargamento di una riga.}
Dobbiamo trovare un indicatore che misuri l'intensità della riga rispetto al continuo adiacente. La quantità di energia che viene sottratta al continuo in questo caso è l'area della conca in Figura \ref{fig:riga-di-assorbimento-generica-nello-spettro}. Questa area è chiamata area equivalente:
\begin{defn}[Area equivalente di una riga $A( \lambda ) $]{def:Area equivalente di una riga}
	Data una riga come in Figura \ref{fig:riga-di-assorbimento-generica-nello-spettro} l'area equivalente della riga è definita come:
\[
	A( \lambda_0) = \int_{\lambda_1}^{\lambda_2} \left( I_{c}-I_{\lambda} \right) d\lambda  
.\]
\end{defn}
Questo parametro è molto importante, infatti decide se saremo in grado di studiare tale riga oppure no: se la riga è troppo sottile può succedere che non si possa risolvere con lo strumento utilizzato per la misura, analogamente se abbiamo righe troppo larghe potremmo non essere in grado di distinguerle dal continuo.\\
Possiamo allora mettere in evidenza quest'ultima affermazione definendo il parametro di larghezza equivalente come:
\begin{defn}[Larghezza equivalente $W( \lambda ) $]{def:Larghezza equivalente}
	La larghezza equivalente è l'Area equivalente normalizzata sull'intensità del continuo:
	\[
	W( \lambda_0) = \frac{A( \lambda_0) }{I_{c}}= 
	\int_{\lambda_1}^{\lambda_2} \left( 1 - \frac{I_{\lambda} }{I_{c}}\right)d\lambda 
.\] 
\end{defn}
\subsubsection{Spessore minimmo di una riga}
\label{subsubsec:Spessore minimmo di una riga}
Il minimo spessore che una riga può avere è quella naturale, quella dovuta al decadimento spontaneo.\\
Prendiamo un atomo a due livelli, la probabilità che l'atomo si disecciti dal secondo livello al primo spontaneamente è proporzionale al coefficiente di Einstein A$_{2,1}$. Quindi la vita media sarà:
\[
	\Delta t = \frac{1}{A_{2,1}} 
.\] 
Ma dal principio di indeterminazione sappiamo che:
\[
	\Delta E\Delta t \ge \hbar
.\] 
Essendo $\Delta t$ finito ci dobbiamo aspettare un allargamento della riga che possiamo quantificare come:
\[
	\Delta \nu = \frac{\gamma _{\text{rad}}}{2\pi}
.\] 
con $\gamma _{\text{rad}}= 1 /\Delta t$, è chiamato Radiative Dumping Costant (RCT).\\
Nello specifico abbiamo che il profilo della emissione spontanea è dato da una Lorentziana:
\[
	\phi ( \mu ) = \frac{\gamma _{\text{rad}}}{2\pi}\frac{1}{\left( \mu -\mu_0 \right) ^2 + \left( \frac{\gamma _{\text{rad}}}{4\pi} \right) ^2}
.\] 
Per atomi con più livelli di eccitazione il parametro RCT si generalizza nel seguente modo:
\[
	\gamma _{\text{rad}} = \sum_{l}^{\infty} A_{kl} \quad k > l
.\] 
Le larghezze naturale delle righe tipicamente è dell'ordine di del millesimo dell' \AA. Tuttavia quando si guarda uno spettro la larghezza che otteniamo è in genere molto maggiore, di fatto la larghezza naturale è trascurabile.\\
Il motivo è che nelle atmosfere delle stelle ci sono altri processi di allargamento della riga aventi un contributo decisamente più importante dell'allargamento naturale.
\subsection{Processi di allargamento delle righe}
\label{subsubsec:Processi di allargamento delle righe}
I principali processi di allargamento dello spettro delle righe sono:
\begin{itemize}
	\item Allargamento termico.
	\item Allargamento collisionale (detto allargamento per pressione), questo cresce con l'aumento della densità dell'atmosfera.
	\item Allargamento per via di campi elettrici o magnetici.
	\item Allargamento rotazionale, questo per via del fatto che non risolviamo la stella e vi è comunque un effetto Doppler. 
\end{itemize}
\subsubsection{Allargamento termico.}
\label{subsubsec:Allargamento termico.}
Concentriamoci solo sulla velocità radiale degli atomi $v_r$: quella che sta sulla linea  della direzione di vista. La distribuzione di questa all'equilibrio termodinamico locale sarà:
\[
	dn( v_{r}) = n\sqrt{\frac{m}{2\pi kT}} 
	\exp\left( -\frac{m v_{r}^2}{2kT} \right) dv_r
.\] 
con $n$ è il numero di atomi per unità di volume dell'elemento interessato, $m$ la sua massa. Applichiamo adesso l'effetto Doppler alla frequenza:
\[
	\nu = \nu_0\left( 1 + \frac{v_r}{c} \right) 
.\] 
Abbiamo adottato la formula non relativistica dell'effetto Doppler, questo perchè alle temperature stellari gli atomi non si muovono abbastanza veloce da esser considerati relativistici. 
Facciamo un esempio: per l'idrogeno a $10000$ K la velocità termica abbiamo:
\[
	v_{\text{th}}
	=
	\sqrt{\frac{2kT}{m}} 
	\sim  \sqrt{\frac{ 2 \text{ eV}}{1 \text{ GeV/}c^2}} 
	\sim \sqrt{20}\cdot 10^{-4}c 
	\sim  13 \text{km}/\text{s}
.\] 
Osserviamo che per come abbiamo scritto la legge dell'effetto Doppler si ha un blue shift quando $v_r > 0$. Lo shift sarà dato da:
\[
	\Delta \nu 
	=
	\nu_0 \frac{v_r}{c}
.\] 
Utilizziamo quest'ultima per ricavarci la distribuzione in frequenza. Procediamo con il cambio di variabile:
\[
	v_r = \frac{c}{\nu_0}\left( \nu - \nu_0  \right) 
	\implies
	dv_r = \frac{c}{\nu_0}d\nu 
.\] 
Sostituendo quindi per la distribuzione in frequenza:
\[
	dn(\nu ) = n \sqrt{\frac{m}{2\pi kT}} \exp\left( - \frac{mc^2}{2\nu_0^2kT}\left( \nu -\nu_0 \right) ^2 \right) \frac{c}{\nu_0}d\nu 
.\] 
Quindi il profilo di riga diventerà, a causa della agitazione termica:
\[
	\phi ( \nu ) = \frac{1}{\Delta \nu_D\sqrt{\pi} }\exp\left( -\frac{\left( \nu -\nu_0 \right) ^2}{\Delta \nu_D^2} \right) 
.\] 
Con 
\[
	\Delta \nu_D = \frac{\nu_0}{c}\sqrt{\frac{2kT}{m}} 
.\] 
Questo è un profilo di tipo Gaussiano con $\Delta \nu _D$ che è detto allargamento Doppler. Questa larghezza cresce al crescere di $T$ e decresce al crescere di $m$.\\
A causa dei moti convettivi abbiamo anche un allargamento di riga per turbolenza, anche questo tipo di allargamento avrà un profilo Gaussiano, soltanto che non sarà dipendente dalla temperatura bensì dalla velocità di turbolenza.\\
Per tenerne di conto si definisce la seguente quantità 
\[
	\Delta \nu_D = \frac{\nu_0}{c}\sqrt{\frac{2kT}{m} + v_\text{turb.}^2} 
.\] 
Il profilo della riga viene spesso espresso in termini di lunghezza d'oonda, possiamo farlo effettuando il canbio di variabile, basta notare che:
\[
	\frac{\lambda - \lambda_0}{\lambda_0} \approx - \frac{\nu - \nu_0}{\nu_0}
.\] 
allora si ottiene facilmente che:
\[
	\phi 
	=
	\sqrt{\frac{mc^2}{2\pi kT\lambda_0^2}} 
	\exp\left( - 
	\frac{mc^2\left( \lambda -\lambda_0 \right)s^2 }{2kT \lambda_0^2} 
	\right) 
.\] 
Possiamo definire anche le quantità più importanti di questo profilo:
\[
	\sigma_\lambda  = \sqrt{\frac{kT}{mc^2}} \lambda_0
.\] 
\[
	FWHM = \sqrt{\frac{8kT \ln 2}{mc^2}} \lambda_0 
.\] 
\subsubsection{Allargamento per collisioni.}
\label{subsubsec:Allargamento per collisioni.}
In questo caso l'allargamento dipende dalla frequenza con cui avvengono le collisioni:
\[
	\nu _{\text{coll}} = n \sigma _{\text{coll}} v_{\text{coll}}
.\] 
Il profilo che si ottiene dalle collisioni è di tipo Lorentziano (come per l'emissione spontanea), si introduce quindi una una larghezza effettiva $\Gamma = \gamma _{\text{rad}} + 2 \nu _{\text{coll}}$, quindi:
\[
	\phi ( \nu ) = \frac{\Gamma }{4\pi^2}\frac{1}{\left( \nu -\nu_0 \right) + \left( \frac{\Gamma }{4\pi} \right) ^2}
.\] 
Nelle atmosfere stellari sono presenti tutti gli effetti citati sopra. Comunamente la zona centrale è dominata dall'allargamento Doppler, nelle code prevale l'effetto della Lorentziana.\\
Ci dobbiamo aspettare che nelle atmosfere rarefatte l'allargamento collisionale sia piccolo, e viceversa nelle atmosfere dense sarà importante. In questo modo si distinguono le stelle giganti da quelle nane.
\subsection{Magnitudine}%
La magnitudine è una misura della luminosità apparente di una stella, nata da Ipparco nel 2$^o$ secolo AC. La catalogazione di Ipparco in 6 classi di magnitudine è ancora usata oggi, con l'aggiunta di un pò di matematica moderna.\\
Le caratteristiche della catalogazione di Ipparco erano:
\begin{enumerate}
	\item Ordine decrescente di luminosità.
	\item Variazione di luminosità costante tra le 6 classi di luminosità.
\end{enumerate}
A rendere matematica questa classificazione è stato Pogston, egli utilizzò le seguenti considerazioni:
\begin{itemize}
	\item La sensibilita dell occhio è logaritmica
	\item Le stelle appartenenti alla sesta classe sono 100 volte meno luminose di quelle di classe uno.
\end{itemize}
Supponiamo di avere due stelle di luminosità apparenti $l_1$ e $l_2$, siano $m_1$ e $m_2$ le magnitudini apparenti di tali stelle. Per quanto assunto sopra avremo che:
\[
	\frac{l_1}{l_2}
	=
	\left( \frac{1}{100} \right) ^{\left( m_1-m_2 \right)/5}
	=
	10^{-2 /5 \left( m_1-m_2 \right) }
.\] 
Passando ai logaritmi:
\[
	\log \left( \frac{l_1}{l_2} \right) 
	=
	-\frac{2}{5}\left( m_1-m_2 \right) 
.\] 
La definizione moderna di magnitudine è quindi la seguente:
\begin{defn}[Magnitudine]{def:Magnitudine}
	Si definisce mangitudine $m_1$ di una stella relativa alla magnitudine di un'altra stella di riferimento $m_2$:
	\[
	m_1-m_2 = -2.5 \log \frac{l_1}{l_2}
	.\] 
	Dove $l_1$ e $l_2$ sono le luminosità apparenti relative alle due stelle.
\end{defn}

\subsubsection{Stelle di riferimento: misure relative di intensità}
Il modo utile di utilizzare la magnitudine è quello di scegliere una stella come riferimento con magnitudine $m_0$ per poter catalogare in maniera relativa tutte le altre stelle.\\
Se vogliamo ad esempio osservare la magnitudine della stella $m_*$ allora abbiamo:
\[
	m_{*}- m_0 = -2.5 \log \left( \frac{l_{*}}{l_0} \right) 
.\] 
Molti sistemi fotometrici usano come riferimento Vega, tra i quali anche il telescopio spaziale Hubble \footnote{Che usa come supporto anche uno spettro articificiale settabile a piacimento.}. Andiamo nel dettaglio sui sistemi di osservazione.
\subsubsection{Parametri di correzione per sistemi fotometrici.}
\label{subsubsec:Parametri di correzione per sistemi fotometrici.}
I limiti di osservazione di un sistema fotometrico sono:
\begin{enumerate}
	\item Non è mai possibile osservare l'intero spettro di emissione della sorgente.
	\item La sensibilità dello strumento nell'intervallo di frequenze che è possibile misurare non è costante.
\end{enumerate}
Per ovviare al fatto che non siamo in grado di conoscere il flusso di energia per unità di tempo e superficie $f( \lambda ) $ possiamo introdurre dei parametri di correzione che tengono di conto dei limiti della strumentazione e delle modifiche che il mezzo apporta alla sorgente.\\
Immaginiamo di poter osservare in un range di lunghezze d'onda da $\lambda_1$ a $\lambda_2$, immaginiamo inoltre che $f( \lambda ) $ sia il flusso della sorgente prima di entrare in atmosfera, la luminosità che si riesce ad osservare sarà data da:
\[
	l = \int_{\lambda_1}^{\lambda_2} f( \lambda ) T(\lambda) d\lambda   
.\] 
Il termine correttivo $T( \lambda ) $ tiene di conto di diversi effetti:
\[
	T( \lambda ) = R( \lambda ) K( \lambda ) Q( \lambda ) A( \lambda ) 
.\] 
Andiamo a vedere qual'è il ruolo di ciascuno di questi:
\begin{enumerate}
	\item $R( \lambda ) $: Riflettività. Questa è legata alle ottiche dello strumento.
	\item $K( \lambda ) $: Correzione alla risposta cromatica del filtro.
	\item $Q( \lambda ) $: Efficienza quantica del rilevatore. 
		Il rilevatore ha una risposta che dipende da $\lambda $.
	\item $A( \lambda ) $: Correzione sulla trasmissione in atmosfera terrestre.
\end{enumerate}
In conclusione possiamo riscrivere la magnitudine tenendo di conto della forma di $l$:
\[
	m_{*} 
	=
	- 2.5 \log \left( 
	\frac{
	\int_{\lambda_1}^{\lambda_2} 
	f_{*}( \lambda ) T( \lambda ) d\lambda }
	{
	\int_{\lambda_1}^{\lambda_2}  
	f_0( \lambda ) T( \lambda ) d\lambda } 
	\right) +
	m_0
.\] 
È quindi chiaro che per parlare di magnitudine è necessario esplicitare il nome dei parametri fissi che si scelgono per l'osservazione: il filtro utilizzato e la stella di riferimento.\\
Ipotiziamo di osservare una sorgente con due filtri differenti, si otterranno due magnitudini dello stesso oggetto differenti:
\[
	m_{*,i} 
	=
	- 2.5 \log \left( 
	\frac{
	\int_{\lambda_1}^{\lambda_2} 
	f_{*}( \lambda ) K_i( \lambda ) d\lambda }
	{
	\int_{\lambda_1}^{\lambda_2}  
	f_0( \lambda ) K_i( \lambda ) d\lambda } 
	\right) +
	m_{0,i}
.\] 
\[
	m_{*,j} 
	=
	- 2.5 \log \left( 
	\frac{
	\int_{\lambda_1}^{\lambda_2} 
	f_{*}( \lambda ) K_j( \lambda ) d\lambda }
	{
	\int_{\lambda_1}^{\lambda_2}  
	f_0( \lambda ) K_j( \lambda ) d\lambda } 
	\right) +
	m_{0,j}
.\]
Possiamo deinire la differenza di magnitudini apparenti come:
\begin{defn}[Indice di colore]{def:Indice di colore}
	L'indice di colore è la differenza di magnitudine apparente misurata con due filtri diversi.
	\[
		I = m_{*,i} - m_{*,j}
	.\] 
\end{defn}
Questo indice ci da una indicazione della temperatura della stella, infatti facendo la differenza tra le magnitudini in questione otteniamo il logaritmo del rapporto tra i flussi in due bande differenti. 
Proprio per questo tale indice ci da una indicazione della temperatura effettiva. \\
Abbiamo visto che possiamo approssimare lo spettro di una stella come quello di corpo nero, misurare lo spettro con due diversi filtri significa esplorare varie zone dello spettro:
\begin{figure}[H]
    \centering
    \incfig{significato-dell-indice-di-colore}
    \caption{Significato dell indice di colore}
    \label{fig:significato-dell-indice-di-colore}
\end{figure}
\noindent
Il rapporto tra le due aree colorate è proprio l'indice di colore. Vediamo dal grafico che tale indice può essere usato anche per una calibrazione dello strumento nella temperaura della stella osservata, infatti è evitende che il rapporto tra le aree vari al variare della temperatura grazie alla legge di spostamento di Wien.\\
\subsubsection{Magnitudine Bolemica.}
\label{subsubsec:Magnitudine Bolometrica.}
Si definisce magnitudine bolometrica la magnitudine in cui si raccoglie l'intero flusso proveniente dalla stella. Per ottenerla è necessario misurare la magnitudine in varie bande e mettere insieme i risultati.\\
La relazione che lega la magnitudine bolometrica a quella in una certa banda è la seguente:
\[
	m_{\text{bol}}= m_{i}+ BC_i
.\] 
In cui abbiamo aggiunto la correzione bolometrica $BC_i$ che dipende dalla banda che stiamo utilizzando.
\subsubsection{Magnitudine Assoluta.}
\label{subsubsec:Magnitudine assoluta.}
Si definisce magnitudine assoluta $M$ la magnitudine apparente che si vedrebbe se la stella fosse distante 10 pc.\\
Prendiamo il flusso $f$ da una stella ad una certa distanza $d$ e con luminosità intrinseca $l$, il flusso sarà dato da:
\[
	f = \frac{l}{4\pi d^2}
.\] 
Se la stella fosse a 10 pc si avrebbe:
\[
	f = \frac{l}{4\pi \left( 10 \text{ pc} \right) ^2}
.\] 
È quindi possibile calcolare le magnitudini associate ai due flussi:
\[\begin{aligned}
	m-M 
	=&
	-2.5 \log \frac{f}{f_{10}}=\\
	=& 
	-2.5 \log \left( \frac{10\text{ pc}}{d} \right) ^2 =\\
	=& 
	- 5 + 5  \log \left( d( \text{pc})  \right) 
.\end{aligned}\]
L'ultima quantità a destra dell'uguale prende il nome di modulo di distanza. \\
Abbiamo assunto che l'unica causa di diluizione del flusso sia la distanza, quindi assumiamo che la radiazione si propaghi nel vuoto. Per correggere e tener di conto dell'assorbimento della radiazione dovuta al mezzo interstellare è necessario aggiundere un fattore alla equazione:
\[
	m-M = -5+5\log d( \text{ pc}) + A_{V}
.\] 
Non è banale ottenere la magnitudine assoluta, infatti in generale è difficile valutare sia $d$ che $A_V$.\\
Si può definire anche la magnitudine bolometrica assoluta che sarà legata alla luminosità intrinseca della sorgente.
