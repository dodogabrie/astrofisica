\lez{11}{30-03-2020}{}
\subsection{Energia delle stelle: dal sole a carbone alle reazioni termonucleari}
\label{subsec:Energia delle stelle: dal sole a carbone alle reazioni termonucleari}
Dai primi dell'800 gli scienziati si indagarono su quale fosse la fonte di energia dalla quale il sole e le altre stelle attingono per brillare nel cosmo senza esaurirsi in tempi generazionali. Vediamo il percorso intellettuale che portò alla conclusione che nel sole avvengono reazioni termonucleari.\\
Durante la rivulozione industriale si avanzava l'idea che il sole potesse brillare bruciando carbone, a smentire tale supposizione ci pensò Mayer. Supponendo che il sole bruci per combustione chimica possiamo chiederci per quanto tempo il sole possa bruciare. Già all'epoca sapevano che con un combustione si libera circa
\[
	E_\text{comb} \sim 4 \cdot 10^{12} \text{ erg/g}
.\] 
Allora il tempo di combuzione di un oggetto di massa $M_{\odot} \sim 2 \cdot 10^{33} $ g e luminosità $L_{\odot} \sim 4 \cdot 10^{33} $ erg/s sarà:
\[
	t_\text{burn} = \frac{E_\text{comb} M_{\odot}}{L_{\odot}} \approx 10^{4} \text{ yr} 
.\] 
Di conseguenza non è possibile che l'energia prodotta dal sole sia originata dalla combuzione chimica, già a quei tempi sapevano che l'età del sole era molto maggiore dl migliaio di anni.\\
Adesso possiamo dire, grazie alla radiodatazione dei meteoriti, che il sole ha una età di circa $T_{\odot} \approx 4.57 $ Gyr. Una sorgente che brucia per tutto questo tempo dovrà avere una energia
\[
	\frac{T_{\odot}L_{\odot}}{M_{\odot}} \sim 2 \cdot 10^{17} \text{ erg/g}
.\] 
Che è una energia molto maggiore di quella che è possibile fare attraverso la combustione chimica.\\
Allora avanzò l'ipotesi che l'energia provenisse dalla gravità, inizialmente si ipotizzo che oggetti come pianeti e meteoriti cadessero sul sole rilasciando energia, questa ipotesi venne presto scartata in favore dell'autogravità.\\
La scorsa lezione abbiamo visto come l'autogravità possa competere con la pressione del gas all'interno del sole, tuttavia non è ancora sufficiente a spiegare il tempo di vita della nostra stella. \\
Dopo quasi un secolo si arrivò alla conclusione che le stelle all'interno delle stelle avvengono reazioni termonucleari, queste sono le responsabili dell'equilibrio su scala di miliardi di anni del sole.
\subsection{Tempo scala nucleare.}
\label{subsec:Tempo scala nucleare.}
Un oggetto della massa del sole avrebbe idealmente una energia disponibile
\[
	E_{\odot}=M_{\odot}c^2 \approx 2 \cdot 10^{54} \text{erg}
.\] 
Questa energia è impossibile da produrre, l'unico modo che si ha per produrla sarebbe quella di far schiantare il sole contro un antisole. Questo non è il nostro caso, nel nostro caso l'energia arriva dalla fusione dell'idrogeno in elio:
\[
	\left( 4m_{H}-m_{He} \right) c^2 \approx 26.73 \text{ Mev} \approx 0.7 \% E_{\odot}
.\] 
Possiamo allora definire il tempo scala nucleare come:
\[
	\tau_{N} = \frac{0.7\%E_{\odot}}{L_{\odot}} \approx 10^{11} \text{ yr}
.\] 
Questo tempo è molto maggiore di quello di Kelvin Halmotz ricavato nella lezione precedente e si avvicina molto al valore noto oggi. Abbiamo allora la seguente disuguaglianza:
\[
	\tau_{FF} \ll \tau  _{KH} \ll \tau_{N}
.\] 
Ipotizziamo adesso che al centro della nostra stella vi sia un gas perfetto
\[
	PV=nkT = \frac{\rho }{\mu m_H}kT
.\] 
\subsubsection{La fase di combustione dell'idrogeno è la più duratura}
\label{subsubsec:La fase di combustione dell'idrogeno è la più duratura}
In risposta alle fusioni la pressione diminuisce poichè diminuisce il numero di particelle quindi avremo una contrazione.\\
Inoltre abbiamo che nella fase più avanzata di fusione che una stella può raggiungere si avrà la fusione di idrogeno a formare il ferro. Quest'ultima fase è quella che fornisce più energia, questa energia è poco maggiore di quella prodotta nel formare elio ($0.9\% E_{\odot}$), a fronte di una grande quantità di atomi di idrogeno necessarri alla formazione dle ferro. Questo implica che la pressione cambierà più repentinamente rispetto al caso dell'elio, quindi sarà una fase meno duratura.\\
Se ne conclude che la fase più lunga della vita di una stella sia quella in cui le reazioni termonucleare coinvolgono la fusione dell'Idrogeno in Elio.
\subsubsection{Termostato stellare}
\label{subsubsec:Termostato stellare}
Le reazioni termonucleari per una stella dipendono anche strettamente dalla temperatura.
Questo aspetto è molto importante infatti sarà necessario che tali reazioni si "regolino" sulla luminosità della stella per controbilanciare la perdita di energia.
\subsection{Equazioni di struttura stellare.}
\label{subsec:Equazioni di struttura stellare.}
Abbiamo visto nelle lezioni precedenti due equazioni che governano la struttura meccanica della stella:
\[\begin{aligned}
	&\frac{\partial P}{\partial r} = - \frac{G m \rho }{r^2}\\
	&\frac{\partial m}{\partial r} = 4\pi r^2\rho 
.\end{aligned}\]
Queste due equazioni contengono le tre variabili del sistema (che ricordiamo essere una shell di un oggetto avente simmetria sferica):
\[\begin{aligned}
	&P=P(r,t)\\
	&m=m(r,t)\\
	&\rho =\rho (m,t)
.\end{aligned}\].
Per descrivere a pieno la struttura di una stella sarà necessario inserire anche una equazione che governa l'energia di quest'ultima, questa sarà una legge di conservazione:
\[
	\frac{\partial L}{\partial r} = 4\pi r^2 \rho \mathcal{E}
.\] 
Tale equazione introduce un'altra variabile: la luminosità della shell $L_1$, questa è la quantità di energia che attraversa la shell di raggio $r$ per unità di tempo. Questa avrà la proprietà:
\[
	L^* = L(r=R)
.\] 
Inoltre abbiamo anche la quantità $\mathcal{E}$  che corrisponde alla quantità di energia prodota da un grammo di materia:
\[
	\mathcal{E}  = \mathcal{E}(\rho , T , \left\{ x_i \right\} ) \quad \text{ erg s}^{-1} \text{ g}^{}
.\] 
Dove $\left\{ x_i \right\} $ indica la composizione chimica. Tale energia può provenire da diversi contributi: quello nucleare, quella trasferita termodinamicamente e quella proveniente da tutti gli altri canali di produzione di neutrini diversi da quello dovuto alla fusione nucleare (questo contributo è già contenuto all'interno  del termine nucleare):
\[
	\mathcal{E}  = \mathcal{E}_N+\mathcal{E}_g +\mathcal{E}_{\nu} 
.\] 
La quantità di energia trasferita termodinamicamente è detta gravitazionale e può essere scritta come:
\[
	\mathcal{E}_g = - \frac{\mbox{d} Q}{\mbox{d} t} = - T \frac{\mbox{d} S}{\mbox{d} t} 
.\] 
L'ultima derivata diventa piccola quando la struttura si evolve con tempi scala dell'ordine di quelli nucleari. Avrà un contributo significativo se la stella evolve con tempi scala dell'ordine di quelli di Kelvin Halmotz.\\
Ipotizzando quindi di essere in una situazione in cui la stella evolve con tempi scala dell'ordine di quello nucleare e che non vi sia produzione di neutrini indipendenti dalle reazioni nucleari, in tal caso allora
\[
	\mathcal{E}  = \mathcal{E}_N
.\] 
Quindi la quantità di energia prodotta dalle reazioni nucleari dovrà essere esattamente uguale ad $L$. Allora l'energia liberata dal sole per radiazioni termonucleari sarà circa $4 \cdot 10^{34} $ erg/s.
\subsubsection{Trasorto di energia nella stella}
\label{subsubsec:Trasorto di energia nella stella}

Nelle stelle abbiamo tre canali di trasporto di energia
\begin{itemize}
	\item Radiativo
	\item Convettivo
	\item Conduttivo
\end{itemize}
Nel caso di trasporto radiativo abbiamo trovato l'equazione per il flusso:
\[
	F = -\frac{4}{3}\frac{ac}{k_R \rho } T^3 \frac{\mbox{d} T}{\mbox{d} r} 
.\] 
Ricordiamo che questa era una equazione di tipo diffusiovo, ottenibile nella approssimazione $l \ll R$.\\
Abbiamo anche visto che all'equilibrio termodinamico si ha
\[
	F = \frac{L}{4\pi r^2}
.\] 
In questo caso particolare possiamo aggiungere alle tre equazioni una quarta che descrive il gradiente di temperatura all'interno di una stella eguagliando le ultime due:
\[
	\frac{\mbox{d} T}{\mbox{d} r} = -\frac{3}{4ac}\frac{k_R\rho }{T^3}\frac{L(r)}{4\pi r^2}
.\] 
Fuori dall'equilibrio radiativo si avrà che il flusso totale no sarà uguale a quello radiativo ma sarà dato dalla somma di tutti i flussi:
\[
	F_\text{tot} =
	\frac{L}{4\pi r^2} 
	=
	F_\text{rad} + F_\text{cond} + F_\text{conv} 
.\] 
Nel caso di LTE domina tuttavia il termine radiativo, vediamo quanto è buona l'approssimazione di LTE all'interno della stella.
\subsubsection{Valutazione dell'LTE}
\label{subsubsec:Valutazione dell'LTE}
Per valutare la bontà della approssimazione di equilibrio termodinamico locale è necessario valutare il gtadiente di temperatura $\nabla T$. Ormai sappiamo che
\[\begin{aligned}
	&T_{\text{centro},\odot} \approx 1.6 \cdot 10^{7} \text{K}\\
	&T_{\text{eff},\odot} \approx 5772 \text{ K}\\
	&R_{\odot} \approx 7 \cdot 10^{10} \text{ cm}
.\end{aligned}\]
Quindi possiamo stimare il gradiente di temperatura come:
\[
	\frac{\Delta T}{R_{\odot}} = \frac{T_c - T_e}{R_{\odot}} \approx 1.7 \cdot 10^{-4} \text{K/cm}
.\] 
Che può esser considerato un buon LTE, nelle zone interne tale valore raggiunge i $10^{-11}$ K/cm, quindi un ottimo LTE. Su lunghezze scala dell'ordine del cammino libero medio le variaizoni di temperatura sono molto piccole rispetto alla temperatura in quel punto.
\subsection{Flusso di energia trasportato dalla conduzione.}
\label{subsec:Flusso di energia trasportato dalla conduzione.}
Anche questa situazione avrà una equazione per il flusso di tipo conduttivo, potremmo allora scrivere che:
\[
	F = - D \frac{\mbox{d} T}{\mbox{d} r} 
.\] 
Dove $D$ sarà uguale a 
 \[
	D = v_e l + cost\ldots
.\] 
Conviene quindi introdurre una opacità conduttiva $k_\text{cond} $ in modo tale da rendere l'equazione del flusso della stessa forma di quella per il trasporto radiativo:
\[
	D = \frac{4ac}{3} \frac{T^3}{k_{\text{cond}}}
.\] 
Se siamo nel caso in cui $F_\text{Rad} + F_\text{Cond} = F_\text{tot}$ allora possiamo scrivere:
\[
	\frac{L}{4\pi r^2} = \frac{4ac}{3}\frac{T^3}{\rho }\left( \frac{1}{k_{\text{cond}} }+ \frac{1}{k_\text{r}} \right)\frac{\mbox{d} T}{\mbox{d} r} 
.\] 
Definendo anche una opacità efficace
\[
	\frac{1}{\overline{k}} =  \frac{1}{k_{\text{cond}} }+ \frac{1}{k_\text{r}}
.\] 
otteniamo un'altra equazione per il gradiente di temperatura:
\[
	\frac{\mbox{d} T}{\mbox{d} r} 
	=
	-\frac{3}{4ac} 
	\frac{\overline{k}\rho }{T^3} \frac{L}{4 \pi r^2}
.\] 
In una stella come il sole (e nella maggioranza delle stelle) si ha che il cammino libero dei fotoni è molto maggiore del cammino libero medio degli elettroni, quindi  si ha anche che:
\[
	k_\text{r} \ll k_\text{cond} 
.\] 
Per via della definizione del coefficiente medio avremo che $\overline{k} \approx k_\text{r} $ come ci si aspetterebbe, infatti l'energia verra trasportata per la maggior parte sotto forma di fotoni.\\
Nelle nane bianche invece la conduzione diventa dominante, infatti queste hanno una densità molto maggiore di quella del sole.
All'aumentare della densità il cammino libero medio diminuisce, d'altra parte però il gas di elettroni diventa degenere all'interno di tali stelle quindi in realtà $l$ cresce. \\
Questo aumento di $l$ è dovuto al fatto che i livelli energetici al di sotto di quelli di fermi sono tutti pieni, quindi non possono interagire, di conseguenza diminuisce anche la probabilità che ogni elettron e ha di interagire e questo comporta un aumento del cammino libero medio.
