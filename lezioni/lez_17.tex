\lez{17}{27-04-2020}{}
Vediamo quando si innesca la combustione dell'${}^{4}$He: il prodotto della combustione dell'idrogeno. \\
Sicuramente dopo la combustione dell'idrogeno il nucleo sarà formato quasi sicuramente da puro elio (i metalli presenti originariamente saranno ancora presenti
\footnote{Se è avvenuta la combustione per CN-NO saranno cambiate le abbondanze relative di questi ultimi}
).\\
Ci aspettiamo che questa combustione inizi ad una temperatura superiore a $T = 15 \cdot 10^8$  K, a questa temperatura avveniva infatti la fusione della fase $p-p$:
\[
    {}^{3}\text{He}+{}^{4}\text{He}\to {}^{7}\text{Be} + \gamma
.\] 
Contrariamente alle nostre aspettative la combustione dell'elio si attiva a temperature molto più grandi dell'ultima citata ($T_\text{He} = 100\cdot 10^6$  K), vediamo oggi perché:
\[
    {}^{4}\text{He}+{}^{4}\text{He} \to {}^{8}\text{Be}
.\] 
Il ${}^{8}$Be è altamente instabile ($\tau_{{}^{8}\text{Be}} \sim 10^{-16}$ s) e decade in due particelle $\alpha$:
\[
    {}^{8}\text{Be} \to {}^{4}\text{He}+{}^{4}\text{He}
.\] 
È necessario fare una piccola parentesi, guardando la tavola periodica possiamo notare che non esistono nuclei stabili con numero di massa $A=5$  oppure $A=8$, infatti anche la reazione 
\[
    {}^{4}\text{He} + p \to {}^{5}\text{Li}
.\] 
Ha un tempo di dimezzamento $\tau_{1 /2} \sim 10^{-22}$ s.\\
Il fatto che non esistano tali nuclei è dovuto unicamente alle proprietà nucleari ed è il motivo per cui il Big Bang non è riuscito a produrre elementi più pesanti del litio, la nucleosintesi cosmologica ha prodotto solo idrogeno per il circa 75\% in massa ed ${}^{4}$He per il 25\%. \\
Oggi sappiamo che i nuclei pesanti si sono formati nelle stelle, la teoria della nucleosintesi stellare è nata negli anni '50 ed è uno dei più grandi risultati in astrofisica.\\
Riprendiamo la reazione all'equilibrio:
\[
    {}^{4}\text{He}+{}^{4}\text{He} \longleftrightarrow {}^{8}\text{Be}
.\] 
Sappiamo che, visto il tempo di dimezzamento del ${}^{8}$Be, l'equilibrio di tale reazione è molto spostato verso sinistra. Aumentando la temperatura sappiamo che il rate delle reazioni di fusione nucleare aumenta (il $\left<\sigma v\right>$ dipende molto sensibilmente dalla temperatura), quindi il tasso della reazione da sinistra a destra aumenta ma, la reazione da destra verso sinistra rimane invariato (non dipende dalla temperatura del gas, il tasso di distruzione rimane inalterato).
Di conseguenza aumentando la temperatura aumenta l'abbondanza di equilibrio del ${}^{8}$Be. \\ 
Per temperature di $T\sim 10^8$ K il ${}^{8}$Be prima di decadere riesce a catturare una particella $\alpha$:
\[
    {}^{8}\text{Be} + \alpha  \to {}^{12}\text{C}+\gamma
.\] 
Abbiamo quindi un processo a 2 step per la formazione del carbonio che prende il nome di $3\alpha$. \\
Se vediamo l'abbondanza di Berillio quando raggiungiamo $T\sim 10^8$ K si ha che:
\[
    \frac{n_\text{Be}}{n_\text{He} } \sim 10^{-10}
.\] 
Quindi nonostante l'aumento della temperatura tale abbondanza rimane comunque molto bassa (capiamo bene quindi perché questo processo non è riuscito ad avvenire durante il Big Bang
\footnote{La grande differenza tra il Big Bang nella fase di nucleosintesi e le stelle è proprio il tempo.}
). \\
Pochi anni dopo ci si accorse però che il processo $3\alpha$ non era sufficiente a spiegare l'abbondanza di carbonio presente nelle stelle perché con l'efficienza calcolata a quel tempo gran parte del carbonio sarebbe stato convertito in ossigeno. Si capì che c'era bisogno di un qualche meccanismo che amplificasse il reaction rate della $3\alpha$ per giustificare tale abbondanza di ${}^{12}$C: doveva esserci un qualche effetto di risonanza.\\
Si scoprì successivamente l'esistenza di un livello eccitato del nucleo di ${}^{12}$C, ad una energia di 7.65 MeV. Tale livello è il motivo della risonanza di cui sopra perché lo stato fondamentale del ${}^{12}$C ha una energia un po più bassa della somma dell'energia del ${}^{8}$Be e del ${}^{4}$He, quindi senza livello eccitato non ci sarebbe stata alcuna risonanza. \\
Viceversa considerando lo stato eccitato vediamo che questo dista dalla somma delle due masse di cui sopra soltanto 300 keV (che tiene di conto della energia cinetica dei due nuclei a temperature dell'ordine di quelle discusse), si crea quindi una risonanza che amplifica enormemente il reaction rate e quindi l'abbondanza finale di carbonio prodotta (di un fattore $10^7$!). \\
Abbiamo quindi che:
\[\begin{aligned}
    &3\alpha  \to {}^{12}\text{C} + \gamma\\
    &{}^{12}\text{C} + \alpha  \to {}^{16}\text{O} + \gamma
.\end{aligned}\]
Per fortuna la seconda reazione non è risonante, altrimenti il ${}^{12}$C dell'universo sarebbe stato svuotato.\\
Per supporre il fatto che la nucleosintesi avvenisse nelle stelle qualche anno prima fu dimostrato, osservando lo spettro di alcune giganti rosse, che si potevano trovare delle righe di assorbimento del Tecnezio (Tc). Le giganti rosse hanno in genere miliardi di anni mentre il Tc ha un tempo di dimezzamento dell'ordine del milione di anni, di conseguenza è necessario che il Tecnezio venga sintetizzato all'interno della stella: non può essere primitivo della stella.\\
Nel 1957 si arriva ad un articolo fondamentale (Burbige,Fouher, Hoyle), una specie di bibbia della nucleo sintesi stellare che include gran parte dei processi che tutt'oggi sono ritenuti corretti.\\
Siamo arrivati alla nucleosintesi dell'ossigeno e del carbonio, il prossimo step richiede temperature dell'ordine di $T\sim 800\cdot $ e porterà alla formazione di magnesio in uno stato eccitato (${}^{24}$Mg$^*$). 
Tale magnesio può decadere in tanti modi diversi: ${}^{12}\text{C}+{}^{12}\text{C}\to $  
\begin{itemize}
    \item ${}^{24}$Mg.
    \item ${}^{20}$Ne + $\alpha$.
    \item ${}^{23}$Na + $p$.
    \item ${}^{23}$Mg + $n$.
\end{itemize}
Questi processi hanno una grossa differenza rispetto ai processi di prima: si incominciano a liberare particelle $\alpha$, $p$, $n$. IL nucleo era rimasto privo di tali elementi, adesso si trova ad avere a temperature così elevate quelle che reagivano già a temperature 800 volte inferiori.\\
Queste particelle reagiranno con i nuclei presenti, oltre alla reazione ${}^{12}\text{C}+{}^{12}\text{C}$ si avranno anche dei network di reazioni dovute a tali particelle.\\
Alla fine di questo macello gli elementi più comuni saranno il ${}^{20}$Ne ed il ${}^{24}$Mg, poi dobbiamo aspettarci anche gli altri isotopi del Neon, il sodio, gli isotopi del magnesio, l'alluminio, il silicio. \\
Finito il ${}^{12}$C la stella si trova con tanto ${}^{16}$O, ${}^{20}$Ne e ${}^{24}$Mg. La stella a questo punto innesca il Neon. \\
La combustione del Neon avviene a temperature dell'ordine del miliardo di gradi, la prima reazione che questo fa è una fotodisintegrazione:
\[
    {}^{20}\text{Ne} + \gamma  \to {}^{16}\text{O}
.\] 
Questo avviene perché a questa temperatura ci sono abbastanza fotoni aventi energia tale da disintegrare ${}^{20}$Ne. Tale elemento ha infatti l'energia più bassa per far avvenire questo processo tra quelli che sono presenti nella stella (4.7 MeV ($\gamma,\alpha$)). Successivamente avviene la reazione:
\[
    {}^{20}\text{Ne} + \alpha  \to {}^{24}\text{Mg} + \gamma
.\] 
La combustione dell'ossigeno avviene per $T\sim 2\cdot 10^9$  K:
\[
    {}^{16}\text{O}+ {}^{16}\text{O} \to {}^{32}\text{S}^*
.\] 
Lo zolfo eccitato a questo punto può fare tante cose:
\begin{itemize}
    \item ${}^{32}$S.
    \item ${}^{28}$Si + $\alpha$.
    \item ${}^{31}$P + $p$.
    \item ${}^{31}$N + $n$.
\end{itemize}
Di nuovo si liberano un sacco di particelle $\alpha$, $n$, $p$. \\
Il prodotto di reazioni più comune sarà il ${}^{28}$Si ed lo ${}^{32}$S. \\
L'ultima catena di combustione (per $T\sim 2.5-3 \cdot 10^9$ K) è la combustione del Si: si ha una combinazione di fotodisintegrazioni, di reazioni delle particelle $\alpha$ create dalle prime e reazioni di fusione. Alla fine della combustione del silicio sarà favorita la formazione degli elementi più legati: quelli del gruppo del ferro. \\
Dopo il ferro siamo nel picco in cui non si ricava più energia dalle reazioni di fusione, è quindi necessario capire come si sono formati gli elementi della tavola periodica che stanno sopra al ferro. L'unico modo che si ha per avere tali elementi è quello di poter fare delle catture neutroniche, servono dei neutroni liberi che possano sintetizzare i nuclei più pesanti. Questo è il motivo per cui si ha una flessione della curva $Z-N$ verso $N$ al crescere di $N$ stesso. Ciò che decide quanti neutroni si possono sintetizzare (nel tempo ) è il flusso di questi ultimi, se il flusso è grande (esplosioni, merge di stelle di neutroni\ldots) allora $N$ aumenta vertiginosamente, successivamente avverranno tanti decadimenti $\beta^+$ fino al raggiungimento del primo nucleo stabile. Si avranno inoltre dei picchi in corrispondenza dei numeri magici nelle curve di abbondanza, tali numeri corrispondono ai nuclei aventi sezioni d'urto di cattura neutronica più basse.
